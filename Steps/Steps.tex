\documentclass{econtex}

\usepackage{econtexSetup}
\usepackage{JobMarket}

\provideboolean{MyNotes}\setboolean{MyNotes}{false}
\opt{private}{\provideboolean{MyNotes}\setboolean{MyNotes}{true}} % Whether to show marginalia; useful for initial meeting with students

\pagestyle{plain}
\begin{document}
\hfill{\tiny \jobname, \today} \vspace{0.1in}

\begin{verbatimwrite}{\jobname.title}
Steps in the Job Market Process
\end{verbatimwrite}
\ifdvi\large\fi
\centerline{\Large Steps in the Job Market Process}\medskip\medskip

\begin{enumerate}
\item Familiarize yourself with this document, \timet, and \faq; in all communications I will expect you
  to have thoroughly digested the information contained in all of the resources
  I have provided.

\item In your initial email to the JMPO and JMCC (see
  \timet~for timing; see \ntn~for definitions of who the JMPO and JMCC are), indicate what your \Moniker~is (don't forget the middle initial!  see \Notation), your main advisor, second advisor, job paper title,
  year in the JHU program, and assessment of likelihood you will
  actually be on the market (for students before 6th year).

\item \ifthenelse{\boolean{MyNotes}}{\marginpar{\tiny Ask computer
      coordinator to help a student set up a listproc for discussions;
      secret from JMPO.}}{} In early September, a login entry will be created for you using your \Moniker~in the
  database that I have set up to keep track of job market candiates,
  \db~ (its full url is at the end of this document).  Don't be
  worried about the information you put in this database; it is mainly
  for internal purposes and we will not (for example) put your phone
  number on the internet.  But we do {\it need} your phone number so that you can be contacted quickly if necessary (in
  response to an inquiry from an employer, e.g.).  {\it In addition,}
  write a short memo about yourself for me and your advisor(s) that
  includes most of the facts you have entered into the database, as
  well as crucial extra content.  In this memo, {\it briefly} describe
  your job market paper topic, sketch the rest of your dissertation,
  mention any other research, and say something about teaching
  interests.  Give a tentative title for your job market paper and
  dissertation (this can change later, in the database).  Say what
  kinds of jobs you are most interested in (e.g.\ university vs.\
  teaching college vs.\ non-academic, large vs.\ small school) and any
  geographic preferences or restrictions.  Note any special selling
  points (e.g.\ good TA ratings), or special connections that could
  result in a job offer outside the usual channels.  Include both
  primary and `backup' contact information so we can always find you
  quickly.  Also include visa status information.  {\it This document
    must not be more than 2 pages.}  Call this document, e.g.,
  \texttt{MemoCarrollCD.pdf}, post it at the root of your \texttt{job
    market directory} (which should correspond to your moniker; see
  \ntn), email it to the JMPO and your dissertation advisers, and
  you're done.  \ifthenelse{\boolean{MyNotes}}{\marginpar{\tiny Think
      about having Nina create a place where they can deposit these
      without my intervention.}}{}

\item Note that you do not want to make an unfavorable impression on
  the job market coordinator by being someone I have to hassle to get
  you to do things like this job market memo.  Please do these things
  on time without my having to pester you.

\item Note that you {\it may} need to pester {\it me} or your advisor to do things
that we are supposed to be doing.  Don't be shy about doing this; for pestering
your advisor, you can blame me and the timetables I have posted (of which all the
faculty are aware).  If you need to pester me, you can remind me of my own
schedules and that I have encouraged you to pester me.

\item Schedule a practice job talk for a date sometime in October in
  one of the department's workshops (if it is impossible to schedule a
  date before the beginning of November, then you can take one later;
  but it is {\it strongly} preferred to do so before the first week of
  November).  (Your presentation must be prepared using the {\it
    Beamer} package for {\LaTeX}.)  Make sure to confirm that your
  advisors can attend on the date of your workshop.  In advance of the
  workshop, recruit one of your fellow grad students to take detailed
  notes on any questions or discussions that arise during your
  seminar.  Also, arrange to make sure that someone is in charge of
  making a video recording of your pitiful performance using the
  department's digital videocam (the JMCC will explain to all of you how this works); seeing yourself on camera will prove
  to be the most effective way to encourage you to correct the cringe-inducing defects in
  your presentation style.
  
\item Get an AEA student membership.  Reliable sources tell us that
  the registration process allows you to indicate areas of interest
  and to somehow indicate that you may be on the job market this year.
  Some recruiters may actually use this information!
  
\item Register to attend the AEA meetings, and try to get a room in
  the hotel where most interviews will take place (the `headquarters'
  hotel).  Don't wait until the last minute, because late
  registrations do not get included in the directory of who is staying
  where, so employers may not be able to find you.  Plan to arrive no
  later than midday or early afternoon on the day before the first
  sessions; in some cases employers might ask to interview you on the
  afternoon of that day.
  
\item Produce a CV (3 pages or less-{\it no exceptions}) and a
  dissertation abstract (using the provided template and following the
  examples of previous generations of students) by {\it early} October.  The
  abstract must be approved by your advisor (and this is a good time
  to discuss with the advisor what you expect will be the contents of
  the entire dissertation).  The abstract must fit on one (1) page
  (that is, it must be fewer than two pages, which is to say that the
  number of pages cannot exceed $e^{-\iota \pi}$); it must use at
  least an 11 point font, and have margins of at least 1.2 inches.  Title at
  top, followed by name, followed by text.  We will do a mass mailing
  of vitas and abstracts to potential employers in mid-October.
  (Include visa/citizenship status on vita).  {\it Use the templates
    provided.}  Post both the PDF and the \texttt{.tex} documents in
  your root \texttt{job market directory}, so others can learn from your pitiful
  \texttt{.tex} efforts and can be slightly less pitiful themselves.

\item Once your CV, abstract, and job market paper are ready, it will
  be {\it your} responsibility to post them into the home page that
  the department will create under your \Moniker.
  Instructions can be found at the \texttt{JMCC Help} web page listed
  below, under the header \texttt{JobMarketComputerHelp}.  To see what
  the final page will look like, go to
  \url{http://www.econ2.jhu.edu/jobmarket/2011/}.  (The page
  \url{http://econ.jhu.edu/research-programs/research-resources/computing-resources/}
  may also contain answers to technical questions.)

\item Produce a finished job market paper by the first week of October.  It should
  be complete, polished, well-written, and nicely formatted.  It
  should look like something ready to send to a journal, not a work in
  progress.  It must be written using {\LaTeX} -- NOT Word, NOT Scientific Word -- {\LaTeX}.  (It's fine to have an
  additional paper or two ready, but quality is {\it much} more
  important than quantity; if you have to make a choice between completing a second or third paper, and 
polishing the job market paper, polish.)
  
\item Around the middle of October, you will begin applying to specific employers
  by sending them a package of materials that I will henceforth refer
  to as your {\Acorn} (contents described below).\footnote{Saying: `From
    The Tiny Acorn, The Mighty Oak Grows' -- a metaphor for your eventual
    job and career, growing from this tiny job packet! (I have coined
    this term to avoid confusion with the `packet' of all JHU CVs and Abstracts that the department compiled and mailed out in former times).}  If you are not
  ready by that time (basically, if you do not have a job market paper
  and your advisors' approval), you will not be on the market.
  
  After you have obtained your advisors' approval for your list of
  places you plan to apply to (and for your job market paper), you
  will follow the procedures in the
  \url{http://econ.jhu.edu/people/ccarroll/jobmarket/RecLetters/}
  directory to get recommendation letters sent on your behalf.

But YOU send your OWN {\Acorn} to each of these employers (either by mail -- for those schools that still accept snail-mail applications -- or electronically).  The {\Acorn} should include
\begin{itemize}
\item A cover letter indicating the job you
  are applying for (remember that sometimes a letter will arrive at
  the wrong place - an economics department in the business school,
  say, or public policy school when it was intended for the School of
  Arts and Sciences - the purpose is to make sure that the letter has
  enough information to get it redirected to the right place)
\item Your CV and dissertation abstract
\item Your job market paper
\item A file folder with your name (e.g., ``Arbatli, Elif C.'') on a
  neatly printed label in the file folder tab.  (This is to endear you
  to the people who will be receiving your application, 90 percent of
  whom would immediately create such a file folder as their first step
  in processing your application, and who will be delighted to have
  been saved 2 minutes of time by your thoughtfulness in preparing the
  folder for them).  The office staff can provide you with pages of
  blank folder labels on which you can print as many copies as
  necessary of your name to stick on the folder tabs.
\end{itemize}

Your letter writers should know long in advance that they will be
expected to write a letter, and you should remind them by, say,
October 23 that their letters will be needed soon.  {\it After} you
have sent your \Acorn out, you can request that your advisers send
their letters.  (It is important to wait: Letters are often lost if
they arrive before an application, because there is no file to put
them in).  

\begin{comment}   % This information should now all be contained in RecLetters, 2012-10-27
Many schools are now accepting letters by email rather than snail
mail.  Nonetheless, often email addresses do not work.  Therefore the
procedure is as follows.  First, compile a list of employers accepting
letters by email, with the appropriate email addresses.  Separately,
you must supply an envelope with the appropriate address on it (these
can be generated using the \texttt{EmployersMoniker.xls} spreadsheet
which you should customize to, e.g.,
\texttt{EmployersArbatliEC.xls}; but please do NOT customize the columns of 
the \texttt{EmployersMoniker.xls} spreadsheet -- this will make it more difficult
for you in generating your letters as well as for the JMPO when trying to figure
out who has applied where).  You must give a printed copy of
your spreadsheet to the office staff, because they will use these
physical paper copies to keep track of their progress in sending out
the letters.  The office will try sending the letters by email.  If
that fails (wrong email addresses are VERY common), the office will
send letters by snail mail.  {\it The department will cover postage on
  any such letters.}  Finally, provide a separate set of envelopes for
the employers who only accept letters by mail.  Your adviser may also
want you to separate potential employers into academic and
non-academic categories; if so, provide separate sets of envelopes for
these categories.
\end{comment}
  
  Your {\Acorns} should go out by 1 week {\it before} Thanksgiving $\mathsf{T}_{-1w}$ (it is
  common for members of hiring committees to bring a pile of applications
  home with them over the Thanksgiving break), and letters should
  follow as quickly as possible thereafter (but cannot be sent until
  you give mailing labels to the dept staff).  

For further information on procedures for getting recommendation letters done, see \recLet.

\begin{comment} % Everyone seems to have abandoned this practice
\item In October and November, a few major employers (IMF, the Fed)
  may schedule preliminary interviews on campus with Hopkins
  students.\footnote{In recent years, these employers seem to have
    reduced or eliminated their campus visits - not just to Hopkins, but in general.
    Don't be disturbed if no Hopkins visit is scheduled.}  Note that
  the IMF tends to like to draw up their list of interviewees themselves
a week or two before they come to campus, which can be problematic
because sometimes they want to do the interviews before the deadline
for deciding whether you are on the market.  This is one of the many
annoying things about the IMF's job application process, because you
may need to make a judgment about applying to the IMF earlier than a
general judgment about whether to be on the market.  However, if you
change your mind after the interview you can withdraw your application
(at some unknown expected cost to your chances for reapplying next
year).
\end{comment} 

\item \ifdvi\hypertarget{Signal}{{\bf [Signal]}~}\fi In the latter half of November, the AEA job registration process 
  allows you to designate no more than two employers to whom you want
  to ``signal'' special interest.  (See the url at the end of this
  document).  These signals, in some cases, actually have value because 
each person has only two of them.  Though you should not obsess about 
where to send your ``signals'' you should give it some thought -- if you
are a plausible candidate for a job at the institution in question, they
will definitely pay more attention to your application if they receive your 
``signal.''  It will single out your application from the 400 others they will have received.
They might receive only 10 ``signals'' so they can afford to pay close attention
to those applications.
  
\item In early December, we will schedule mock interviews in which
  Hopkins faculty pretend to be interviewers at the AEA meetings and
  grill you as you will be grilled there.
  
\item Between early December and late December, employers who are
  potentially interested in you will contact you to schedule an
  interview at the AEA meetings.  It is vital that during this period
  (even over the holidays) you are in touch with your email and phone
  messages.  If you will be away, learn how to retrieve voicemail
  messages from your answering machine before you go!  (Or list as
  your contact phone number a cellphone number that will work wherever
  you may go; note that you CANNOT expect employers to hunt you down -
  the phone number you provide to them should succeed in contacting
  you with 100 percent probability).
  
\item Prepare for the interviews; several students have provided
  detailed advice, which I have posted in the ``Resources'' directory
  of my job market page.  Also, the document
  \texttt{Practice.Interviews.Student.Preparation.doc} provides an
  excellent overview and guidance. One further requirement is to be
  able to articulate an explicit plan for when your dissertation will
  be completed and what it will consist of; many employers, in
  interviews, will ask this question, and you need to have an answer
  (approved by your advisor) ready at hand. \ifdvi\hypertarget{InterviewPrep}{}\fi
  
\item Beginning in late January, and extending to late March,
  employers who are still interested in you will call to schedule a
  job talk.
  
\item After any given job talk, an employer can wait anywhere from a
  day to two months before getting back in touch with you.  But it is
  socially acceptable for you to call or email to ask for an update on
  your status.
  
\item If you don't have a job by late March, don't despair.  In the
  end, experience shows that eventually everyone finds a job!

\item When your final job status is resolved (that is, when you accept 
a job offer), email me, your advisors, and the JMCC to make sure we have the info.

\end{enumerate}
  
One activity that does not fit neatly into the chronological framework
articulated above is the process of developing the list of places you
want to apply to.  You should start this process as soon as your
advisor gives you preliminary approval to be on the market.
Periodically check Job Openings for Economists (JOE) (available from
the AEA) for job listings.  You might also check listings from the
Financial Management Association and the National Association of
Business Economists.  If you are interested in overseas jobs, look at
the European JOE and the ads in the {\it Economist}.  (If you discover
job listings that may be of interest to your classmates, please let
them know).

You will ultimately need to generate form letters to the employers you
plan to apply to, and will need various kinds of information about
them (like email addresses and phone numbers) as you work your way
through the process.  Over the years, we have developed a standardized
Excel spreadsheet template that contains slots for all the information
you will need to collect, available in the \Templates~folder as
\texttt{EmployersMoniker.xls}.  You {\it MUST} use (a renamed version
of) this Excel spreadsheet template when you send out your mailing
(because you want to use the linked form-letter-generating software);
therefore, to avoid confusing yourself by having multiple lists, you
should start your process by downloading the spreadsheet
\texttt{EmployersMoniker.xls} from the templates directory.  Rename
it to, e.g., \texttt{EmployersCarrollCD.xls}, erase the information
that is in it (unless you want to apply to some of the example
employers included in the template file -- in which case, you still
need to verify that they are hiring this year and the contact
information has remained the same).  Please {\it do not} rearrange or
rename the columns in this template -- in the end, the JMPO will need
to merge all of the students' templates in order to have a manageable
idea of who has applied where, and you do not want to piss off the
JMPO at this crucial time by making extra work in the form of figuring
out what you have renamed columns to and where you have moved them to.
(You are welcome to add extra columns beyond those in the template,
and hide those that you don't need to use; but do NOT rename or
reorder them!).

When you have developed a list of places to which you wish to apply, make an appointment with your advisor(s) to discuss the list.  When you and your advisor(s) have agreed on the final list, you will need to give copies of the list to the people who are writing reference letters.  See \recLet~for procedures.  But YOU send the {\Acorn}; some employers also require a transcript, and it probably makes sense to include one in all your {\Acorns} rather than trying to make some {\Acorns} with and some without it.  For academic employers, some will ask for evidence of teaching skills; here, there is no clear-cut standard for what they expect to receive, but student evaluations of your performance (including a few choice quotes from the student evaluation forms) are a good idea.  Do NOT send more than a page or two of info on this subject - nobody will read more than this, so distill everything down to the essence.

Err on the side of too many applications rather than too few.  But
don't waste everyone's time by applying to places that you would turn
down under all possible circumstances.
\ifthenelse{\boolean{MyNotes}}{\marginpar{\tiny Emphasize the point
    that they may not apply to places where they would not accept if
    it were the only offer.  If you have an offer and decline it, then
    you will get no further assistance from the department.}}{}
  
One more point: You should have a regular computer backup plan
(ideally, an automatic backup every night; at a minimum, once a week)
for all the computer files that are critical to your thesis research,
job market paper, and application process.  Without a backup, you are
at risk of losing months worth of work or more, and in a worst case
scenario it could mean you will have to withdraw from the job market.
In 2003 one student's laptop crashed right in the middle of the job
market.  Fortunately the student had a week-old backup of the key
files; but let that be a warning.

\pagebreak

\together{
\begin{center}
\centerline{\Large Electronic Job Market Resources}
\medskip\medskip
\small
\begin{tabular}{rl}
   Overall & \url{http://www.econ2.jhu.edu/people/ccarroll/JobMarket}
%\\ emailing list & \url{mailto:JhuJobMarketPublic@googlegroups.com}
\\ Database      & \url{http://jhuEconPeople.dynalias.org}
\\ Computer Help      & \url{http://econ.jhu.edu/jobmarket/information/JobMarketComputerHelp.html}
%\\ FacGmail List      & \url{http://econ.jhu.edu/jobmarket/information/gmail-accts.pdf}
\\ Cover Letter & \url{http://www.econ2.jhu.edu/jobmarket/Information/CreateLabel\&CoverLetterInstruction.html}
\\ ASSA meeting & \url{http://www.vanderbilt.edu/AEA/anmt.htm}
\\ \JOE & \url{http://www.aeaweb.org/joe}
\\ \JOE Signal & \url{http://www.aeaweb.org/joe/signal/}
%\\ Rumors & \url{http://www.econjobrumors.com/}
%\\ Placements & \url{http://www.econ2.jhu.edu/recent_placements}
\\ JMCC Help  & \url{http://www.econ2.jhu.edu/jobmarket/Information/}
\\ EconJobMarket & \url{http://econjobmarket.org/candidates.php}
%\\ Job Info & \url{http://www.illinoisskillsmatch.com/}
\\ Public Wiki & \url{http://bluwiki.com/go/Econjobmarket}
%\\ Rankings & \url{http://www.econ2.jhu.edu/people/ccarroll/JobMarket/Resources/Rankings.doc}
\\ Advice & \url{http://www.econ2.jhu.edu/people/ccarroll/JobMarket/Resources/}
\end{tabular}
\end{center}}
\normalsize

The Advice category is particularly noteworthy, as it contains (among
other things) a variety of documents specifically written by either
JHU students or JHU faculty with their personal advice about various
aspects of the job market process.  For example, the document
\texttt{AdviceMoffittInterviews.doc} contains advice Robert Moffitt
has provided about how to handle job market interviews.

\begin{comment} % Seems defunct, 2012-11-27
\url{http://www.illinoisskillsmatch.com/} is a site for posting your
resume and other information for employers to browse.  It may or may
not be a valuable supplement to the ASSA process.
\end{comment}

One of you (the students on the job market) should create an emailing
list for you to email each other with gossip, tips, useful urls that
you discover, job listings that others might not have seen, etc.  


\end{document}
